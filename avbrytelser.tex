\section{Avbrytelser}
\label{chp:avbrytelser} 

Sykehus er komplekse organisasjoner som stadig presses til å levere sine tjenester så effektivt som mulig \cite{Scholl07}. Likevel viser observasjon at synkron kommunikasjon, eksempelvis ansikt-til-ansikt eller per telefon, foretrekkes til tross for at dette kan være ineffektivt og forårsaker et avbruddsdrevet miljø \cite{Scholl07, Parker00}. Sykepleiere baserer seg på samarbeid, og avbrudd er dermed en naturlig del av arbeidshverdagen \cite{Bardram04}. 
Disse avbruddene kan ha betydelige, skadelige effekter på pasientsikkerhet, da en økende mengde bevis indikerer at avbrudd er en av de viktigste årsakene til medikasjonsfeil \cite{Parker00, McGillis10}. Arbeidsoppgaver ved sykehus er spredt over tid og sted, og informasjon varierer ofte i kompleksitet og hastegrad. Dette gjør effektiv kommunikasjon utfordrende \cite{KlemetsRedundancy}. 

\noindent
\emph{Kognitiv kapasitet}
Parker og Coiera hevder at en begrensende faktor i enhver kommunikasjonsanalyse er den kognitive kapasiteten individer har til å gjennomføre sitt arbeid, da studier har vist at feil og ineffektivitet er et resultat av at denne kapasiteten er overskredet. Kunnskap om menneskets hukommelse hevdes å være nøkkelen til å forstå hvilke krav som bør settes til teknologi brukt i slike omgivelser \cite{Parker00}.
Det skilles vanligvis mellom langtids- og korttidsminne. Den passive kunnskapen man besitter ligger i langtidsminnet, for eksempel medisinske fakta eller viktige datoer, mens kortidsminnet, eller abeidsminnet, er den bevisste delen av minnet som aktivt behandler informasjon. Arbeidsminnet har begrenset kapasitet og varighet, og lar seg raskt forstyrre av distraksjoner og avbrytelser. Coiera og Tombs antyder at synkron kommunikasjon foretrekkes fordi det gir en umiddelbar bekreftelse på at en beskjed er mottatt. Dersom man ønsker å gi en beskjed eller et ansvar videre, vil denne usikkerheten bli liggende i arbeidsminnet frem til man får en bekreftelse fra mottaker. Sykepleiere som arbeider i avbruddsdrevne omgivelser er dermed utsatt for svikt i arbeidsminnet dersom denne kapasiteten overstiges \cite{Parker00}. 

\noindent
Grundgeiger og Sanderson skiller mellom «gode» og «dårlige» avbrudd, og hevder disse bør sees i sammenheng med hvilke effekter de har. Evjemo og Klemets kaller dette dualiteten ved avbrudd, personen som forårsaker avbruddet opplever å få en umiddelbar bekreftelse på at informasjonen er mottatt og kan dermed avlaste arbeidsminnet, mens den som blir avbrutt kan oppleve en negativ effekt, eksempelvis at den kognitive kapasiteten overskrides, forsinkelse i eget arbeid, stress og frustrasjon. Samtidig kan avbruddet ha en positiv effekt, ved at den som blir avbrutt mottar en alarm om en pasients alvorlige tilstand, eller mottar ønsket informasjon \cite{Evjemo, Grundgeiger09}. 
